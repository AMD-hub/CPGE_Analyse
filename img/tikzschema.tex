%%%%%%%%%%%%%%
\tikzstyle{deduct} = [rectangle, rounded corners, minimum width=3cm, minimum height=1cm,text centered, draw=black, fill=red!30]
\tikzstyle{test} = [rectangle, minimum width=3cm, minimum height=1cm, text centered, draw=black, fill=orange!30]
%%%%%%%%%%%%%%%%%
\resizebox{\linewidth}{!}{
\begin{tikzpicture}

%Nodes

%%etage0 :
\node[test](maintopic){ $u_n \longrightarrow 0 \ ? $ };
\node[deduct](rightsquare)[right=of maintopic]{\diverge };
%%etage1 :
\node[test](maintopic1)[above=of maintopic]{$u_n \geq 0$ };
\node[test](rightsquare1)[right=of maintopic1]{$ \sum |u_n| $  conv ? };
\node[test](rightsquare12)[right=of rightsquare1]{$ u_n $  est alterné ? };
\node[deduct](rightsquare13)[right=of rightsquare12]{Utiliser DA };

%%etage2 :
\node[deduct](rightsquare2)[above=of rightsquare1] {\conv };
\node[deduct](rightsquare22)[above=of rightsquare12] {Utiliser CSSA };

%%etage3 :
\node[test](maintopic3)[above=of maintopic , yshift=4cm]{on choisit selon l'exo};
\node[deduct](leftsquare3)[left=of maintopic3 ]{  régles Cauchy / D'alembert  };

%%etage4 :
\node[deduct](maintopic4)[above=of maintopic3 ]{  regles de comparaisons  };
\node[deduct](rightsquare4)[right=of maintopic4]{comparaison avec integrale} ;

%Lines
%etage0 :
\draw [-] (maintopic) -- node[anchor=south] {no} (rightsquare);
\draw [-] (maintopic) -- node[anchor=east] {yes} (maintopic1);
%etage1 :
\draw [-] (maintopic1) -- node[anchor=south] {no} (rightsquare1);
\draw [-] (rightsquare1) -- node[anchor=east] {yes} (rightsquare2);
\draw [-] (rightsquare12) -- node[anchor=east] {yes} (rightsquare22);
\draw [-] (rightsquare1) -- node[anchor=south] {no} (rightsquare12);
\draw [-] (rightsquare12) -- node[anchor=south] {no} (rightsquare13);
\draw [-] (rightsquare22) -| node[anchor=south] {\hspace{-2.5cm} \begin{small} ne marche pas\end{small}} (rightsquare13);
\draw [-] (maintopic3) -|  (rightsquare13);
\draw [-] (maintopic3) -|  (rightsquare4);

\draw [-] (maintopic1) -- node[anchor=east] {yes} (maintopic3);
\draw [-] (maintopic3) --  (maintopic4);
\draw [-] (maintopic3) --  (leftsquare3);
\end{tikzpicture}
		 				}