\documentclass[../Main.tex]{subfiles}

\begin{document}

\chapter{LES TECHNIQUES DE CALCUL DES INTÉGRALES ET DES SÉRIES}

\section{Techniques de base}

\subsection{Pour les intégrales}
Les méthodes de base sont des techniques fondamentales pour résoudre des intégrales simples. Elles reposent sur des manipulations directes de l'intégrande pour simplifier le calcul.

\meth{Addition}{
Utilisée lorsque l'intégrande est une somme de termes intégrables individuellement.
Soient \( f, g : ]a, b[ \rightarrow \mathbb{R} \) deux fonctions de classe \( \class1 \) telles que $\int_{]a,b[} f(x)dx$  et $\int_{]a,b[} g(x)dx$ convergent, alors la formule est :
\[
\int_{]a,b[} (f(x) + g(x)) \, dx = \int_{]a,b[}f(x) \, dx + \int_{]a,b[} g(x) \, dx.
\]
}
\exm{}{
Calculons \( \int (x^2 + 3x) \, dx \).
\sol
En séparant les termes, on obtient
\[
\int (x^2 + 3x) \, dx = \int x^2 \, dx + \int 3x \, dx.
\]
Chaque terme est ensuite intégré séparément :
\[
= \frac{x^3}{3} + \frac{3x^2}{2} + C,
\]
où \( C \) est la constante d'intégration.
}

\meth{Intégration par parties}{
Basée sur la formule \( \int u \, dv = uv - \int v \, du \), cette méthode est utile lorsque l'intégrande est le produit de deux fonctions dont l'une est facilement dérivable et l'autre intégrable.\\
Soient \( f, g : ]a, b[ \rightarrow \mathbb{R} \) deux fonctions de classe \( \class1 \) telles que \( \lim_{t \to a} f(t)g(t) \) et \( \lim_{t \to b} f(t)g(t) \) existent. Alors les intégrales \( \int_a^b f(t)g'(t) \, dt \) et \( \int_a^b f'(t)g(t) \, dt \) sont de même nature. Lorsqu'elles sont convergentes, on a
\[
\int_a^b f'(t)g(t) \, dt = f(b)g(b) - f(a)g(a) - \int_a^b f(t)g'(t) \, dt.
\]
}
\exm{}{
Calculons \( \int_{0}^{+\infty} x e^{-x} \, dx \).
\sol
Choisissons les fonctions :
\[
\begin{cases}
    u(x) &= x \\ 
    v^\prime(x) &= e^{-x}
\end{cases}
\Longrightarrow
\begin{cases}
    u^\prime(x) &= 1 \\ 
    v(x) &= -e^{-x}
\end{cases}
\]
On vérifie aisément que :
\begin{itemize}
    \item Ils s'agissent sont de classe $\class^1$,
    \item Les limites existent : $\lim_{x\rightarrow 0} u(x)v(x)dx = 1$ et $\lim_{x\rightarrow +\infty} u(x)v(x)dx = 0$
    \item La convergence de $\int_0^{+\infty} u^\prime(x)v(x)dx$
\end{itemize}
Donc on peut appliquer la formule,
\[
\int_0^{+\infty} x e^{-x} \, dx = \left[ -x e^{-x}\right]_0^{+\infty} - \int_0^{+\infty} -e^{-x} \, dx.
\]
On obtient finalement
\[
\int_0^{+\infty} x e^{-x} \, dx = 1
\]
}

\meth{Changement de variable}{
Consiste à transformer l'intégrale avec une nouvelle variable pour simplifier l'intégrande. \\ 
Soit \( f \) une fonction continue sur \( ]a, b[ \) et \( \varphi : ]\alpha, \beta[ \rightarrow ]a, b[ \) bijective, strictement croissante et de classe \( \class^1 \). Les intégrales \( \int_a^b f(t) \, dt \) et \( \int_\alpha^\beta f \circ \varphi(u) \varphi'(u) \, du \) sont de même nature et en cas de convergence on a :
\[
     \int_a^b f(t) \, dt  = \int_\alpha^\beta f \circ \varphi(u) \varphi'(u) \, du 
\]
}
\exm{}{
Calculons \( \int_0^{+\infty} x e^{-x^2} \, dx \).
\sol
Posons, $x = \phi(u) = \sqrt{u}$, on a $\phi : ]0,+\infty[ \mapsto ]0,+\infty[$ 
 est une fonction bijective, croissante et de classe $\class^1$. De plus on a :
Donc, la nature de l'intégrale   $ \int_0^{+\infty} x e^{-x^2} \, dx$ est celle de l'intégrale : $ \int_0^{+\infty} x e^{-u} \frac{1}{2\sqrt{u}} \, du = \int_0^{+\infty} \frac{1}{2}e^{-u}  \, du$ et de plus :
\[
\int_0^{+\infty} x e^{-x^2} \, dx =  \int_0^{+\infty} x e^{-u} \frac{1}{2\sqrt{u}} \, du = \int_0^{+\infty} \frac{1}{2}e^{-u}  \, du = 1
\]
}


\section{Méthode de Feynman}

\subsection{Application aux intégrales}
La méthode de Feynman permet de simplifier le calcul d'intégrales complexes en introduisant un paramètre supplémentaire pour obtenir une formulation paramétrique.

\meth{Méthode de Feynman pour les intégrales}{
Pour calculer une intégrale de la forme $\int_{a}^{b} f(u) \, du$, on peut définir une fonction paramétrée comme suit : 
\[
F(x) = \int_{a}^{b} \varphi(u, x) \, du
\]
L'objectif est ensuite d'étudier la fonction $F(x)$ en trouvant, par exemple, une équation différentielle ordinaire (EDO) dont elle est solution. En résolvant cette EDO, on obtient une expression pour $F(x)$, permettant ainsi de déduire une expression de l'intégrale initiale. \\
Les choix célèbres de la fonction $\varphi(u, x) $ sont :
\begin{itemize}
    \item \textbf{Transformé de Laplace : } $\varphi(u, x) = f(x)e^{-xu}$ 
    \item \textbf{Transformé de Fourier : } $\varphi(u, x) = f(x)e^{-ixu}$ 
    \item \textbf{Dilatation de variable : } $\varphi(u, x) = f(x.u)$ 
\end{itemize}
}

\exm{Intégrale de Gauss}{
Calculer l'intégrale de Gauss :
\[
\int_0^{+\infty} e^{-x^2}dx
\]
\sol
Considérons,
\[
f(t) = \int_0^{\infty} \frac{e^{-t^2(1 + x^2)}}{1 + x^2} \, \mathrm{d}x
\]
La dérivée de \( f(t) \) par rapport à \( t \) est [nous laissons la vérification des hypothèses pour le lecteur] :
\[
f'(t) = \int_0^{\infty} \frac{-2t e^{-t^2(1 + x^2)} (1 + x^2)}{1 + x^2} \, \mathrm{d}x
\]
ce qui se simplifie en :
\[
f'(t) = -2t e^{-t^2} \int_0^{\infty} e^{-t^2 x^2} \, \mathrm{d}x
\]
Posons \( u = tx \), ce qui donne \( \mathrm{d}u = t \, \mathrm{d}x \), où \( t \) est une constante.
En remplaçant dans l'équation, on obtient :
\[
f'(t) = -\frac{2t e^{-t^2}}{t} \int_0^{\infty} e^{-u^2} \, \mathrm{d}u
\]
Nous savons que
\[
I = \int_0^{\infty} e^{-u^2} \, \mathrm{d}u
\]
Par conséquent,
\[
f'(t) = -2 e^{-t^2} I
\]
où \( I \) est une constante.
Maintenant, en intégrant les deux côtés avec les bornes de \( 0 \) à \( \infty \), on a :
\[
\int_0^{\infty} f'(t) \, \mathrm{d}t = \int_0^{\infty} -2 e^{-t^2} I \, \mathrm{d}t
\]
\[
\int_0^{\infty} f'(t) \, \mathrm{d}t = -2I \int_0^{\infty} e^{-t^2} \, \mathrm{d}t
\]
Ainsi,
\[
f(\infty) - f(0) = -2I^2 \quad \text{(Équation (2))}
\]

Calculons maintenant les valeurs limites de \( f(t) \) :

\[
f(\infty) = \int_0^{\infty} \frac{e^{-\infty \cdot (1 + x^2)}}{1 + x^2} \, \mathrm{d}x = 0
\]
\[
f(0) = \int_0^{\infty} \frac{1}{1 + x^2} \, \mathrm{d}x  = \frac{\pi}{2}\quad [\textit{C'est un arctangente}]
\]
}

\subsection{Application aux séries}
Cette méthode peut également être étendue aux séries pour faciliter le calcul de certaines sommes en introduisant un paramètre supplémentaire.

\meth{Méthode de Feynman pour les séries}{
Pour calculer une somme de la forme $\sum u_n$, on peut considérer une fonction paramétrée de la série :
\[
F(x) = \sum u_n(x)
\]
L'objectif est ensuite d'étudier la fonction $F(x)$, souvent en trouvant une série entière associée ou, si possible, une EDO. En résolvant cette équation, on peut alors obtenir une expression pour $F(x)$ et, par conséquent, pour la somme initiale.\\
Les choix célèbres de la série des fonctions $u_n(x) $ sont :
\begin{itemize}
    \item \textbf{Série entière : } $u_n(x) = u_nz^n$ 
    \item \textbf{Séries de Fourier [Hors programme] : }  $u_n(x) = u_ne^{inx}$  
\end{itemize}
}

Vous trouverez plus des exemples dans la partie des exercices.


\section{Méthode de AMD}
C'est un truque qui consiste à changer une intégrale en une série ou une série en une intégrale à l'aide des théorèmes d'inversion \textit{Intégrales - Somme}. 

\meth{Méthode de AMD}{
\textbf{Cas des intégrales : } Pour calculer un intégrale $\int_{]a,b[} f(u)du$ on essaie d'exprimer $f$ en somme d'une série des fonctions  $\sum f_n$ puis on utilise une formule d'inversion (\textbf{après avoir vérifier les conditions d'application}) pour déduire :
\[
\int_{]a,b[} f(u)du = \sum \int_{]a,b[} f_n(u)du 
\]
De telle façons que la série résultant est plus facile à calculer. \\
\textbf{Cas des séries : } Analogiquement, et c'est le cas le plus pratique, on peut pour calculer une série $\sum_n u_n$ exprimer les termes $u_n$ en une intégrale $\int_{]a,b[} f_n(u)du$ et puis on utilise une formule d'inversion (\textbf{après avoir vérifier les conditions d'application}) pour déduire :
\[
\sum_n u_n = \int_{]a,b[} \left(\sum  f_n(u)\right)du 
\]
On pourra penser à utiliser les remplacements suivants:
\begin{align*}
\frac{1}{n} &= \int_0^1 t^{n-1} \, dt\\
\frac{1}{\alpha n + \beta} &= \int_1^{\infty} t^{\alpha n + \beta -1} \, dt \\ 
\frac{1}{n^\alpha} &= \int_0^{\infty} t^{\alpha} e^{-\alpha t} \, dt\\
n! &= \int_0^{\infty} t^n e^{- t} \, dt
\end{align*}
}


\exop{}{
Calculer la série :
\[
\sum_{n=1}^{+\infty} \frac{\cos(n\theta )}{n}
\]
}{

Soit \(\theta \in \mathbb{R}\). calculons pour \(\theta \in \mathbb{R}\), \(\theta \neq 0\) \([2 \pi]\), la somme :

\[ S(\theta) = \sum_{n=1}^{+\infty} \frac{\cos(n\theta)}{n} \]

Soit \(\theta\) une telle réelle. L'idée consiste à transformer :
\[ \frac{\cos(k\theta)}{n}  \quad \textit{vers}\quad \int_0^1 x^{n-1} \cos(n\theta) \, dx\]
Et donc :
\[ \sum \frac{\cos(k\theta)}{n}  \quad \textit{vers}\quad \sum \int_0^1 x^{n-1} \cos(n\theta) \, dx\].
Et si on peut inverser intégrale-somme (c'est ce qu'on va voir) on obtiendra :
\[ \sum_{n=1}^{+\infty}  \frac{\cos(k\theta)}{n}  \quad \textit{vers}\quad \int_0^1 \sum_{n=1}^{+\infty}  x^{n-1} \cos(n\theta) \, dx\]
Ainsi on sera obligé de calculer \(\sum_{n=1}^{+\infty} x^{n-1} \cos(n\theta)\). Ce calcul est plus facile en utilisant les complexes.   En effet :
\[
\sum_{n=1}^{+\infty} x^{n-1} \cos(n\theta) = \Re \left( \sum_{n=0}^{+\infty} x e^{i \theta} \right) = \Re \left( \frac{e^{i\theta}}{1 - x e^{i \theta}} \right)
\]

Pour la rédaction, on commencera par poser :
\[
C(x) = \sum_{n=0}^{+\infty} x^{n} e^{i (n + 1) \theta} = \frac{e^{i\theta}}{1 - x e^{i\theta}}
\]
Cette série est bien définie sur \([0,1[\). On va appliquer le théorème de convergence terme à terme \ref{thm:CTT} 
\begin{enumerate}
\item pour tout entier, $n$ la fonction $f_n : x\mapsto x^{n} e^{i (n + 1) \theta} $ est continue par morceaux sur $I$.
\item $\sum_n f_n \xrightarrow{CVS \text{ sur } I } C$ avec $C$ une fonction définie sur $[0,1[$ et continue par morceaux sur cet intervalle.
\item \(\sum \int_0^1 |x^n e^{i(n+1) \theta}| \, dx = \sum \frac{1}{n + 1}\) diverge
\end{enumerate}
Donc on peut par appliquer cette théorème!!!, nous essayons une autre théorème, la théorème de convergence dominé \ref{thm:CDSerie}:
\begin{enumerate}
\item pour tout entier, $n$ la fonction $f_n$ est continue par morceaux sur $[0,1[$.
\item $\sum_n f_n \xrightarrow{CVS \text{ sur } [0,1[ } C$ avec $C$ une fonction définie sur $[0,1[$ et continue par morceaux sur cet intervalle.
\item On a : 
\[
S_{N} = \sum_{n=0}^{N-1} x^n e^{i(n+1)\theta} = C(x)\left( 1-x^Ne^{N\theta}\right) 
\]
Donc :
\[
|S_N| \leq \left|C(x)1+x^Ne^{N\theta}\right| \leq |C(x)||1+x^Ne^{N\theta}|\leq |C(x)|\left(|1|+|x^Ne^{N\theta}|\right) \leq 2|C(x)| 
\]
Il existe donc une fonction $\phi$ (à savoir $=2|C(x)|$ ) continue par morceaux et intégrable sur $[0,1[$ tq : 
\[ \forall N \in \NN \text{ \ \ : \ \ }  \left| \sum_{k=0}^{N} f_n(x) \right| = |S_N| \leq \phi(x) \ \forall x \in I\]
\end{enumerate}
On a le résultat suivant : 
\begin{itemize}
\item $C$ et $f_n$ sont intégrables sur $[0,1[$ pour tout entier $n\in\NN$.
\item La série $\displaystyle \sum_n \int_{0}^1 f_n(x)dx  $ converge et La formule d'inversion : 
\end{itemize}
\begin{equation}
\sum_{n=0}^{+\infty} \int_{0}^1 f_n(x)dx = \int_{0}^1 \left( \sum_{n=0}^{+\infty} f_n(x) \right) dx  = C(x)
\end{equation} 
Nous avons aboutit à notre objective :
\begin{align*}
    \sum_{n=1}^{+\infty}  \frac{\cos(n\theta)}{n}
    &= \Re\left(   \sum_{n=1}^{+\infty} \frac{e^{in\theta}}{n}  \right) \\
    &= \Re\left(   \sum_{n=1}^{+\infty} \int_0^1 x^{n-1} e^{in\theta}dx  \right) \\
    &= \Re\left( \int_0^1   \sum_{n=1}^{+\infty} x^{n-1} e^{in\theta}dx  \right) \quad \textit{Par inversion} \\
    &= \Re\left( \int_0^1 \frac{e^{i\theta}}{1 - x e^{i\theta}} dx\right) \quad \textit{Déjà calculé.}\\
    &=  - \lno{|2 \sin(\frac{\theta}{2})|}  \quad \textit{Laissé au lecteur}
\end{align*} 
}

\newpage
\section{Résumé de probabilité }
Nous donnons un résumé des formules probabilités utile (c'est pas un cours, il faut lire votre cours)



\subsection{Espérance}
\paragraph{Cas Discret}
Soit \( X \) une variable aléatoire discrète prenant des valeurs \( x_i \) avec des probabilités \( p_i = P(X = x_i) \).

\[
\mathbb{E}[X] = \sum_{i} x_i \cdot p_i
\]

\paragraph{Cas Continu}
Soit \( X \) une variable aléatoire continue avec une densité de probabilité \( f(x) \).

\[
\mathbb{E}[X] = \int_{-\infty}^{+\infty} x \cdot f(x) \, dx
\]

\subsection{Formule de Transfert}

\paragraph{Cas Discret}
Pour une fonction \( g(X) \) d’une variable discrète \( X \) :

\[
\mathbb{E}[g(X)] = \sum_{i} g(x_i) \cdot p_i
\]

\paragraph{Cas Continu}
Pour une fonction \( g(X) \) d’une variable continue \( X \) :

\[
\mathbb{E}[g(X)] = \int_{-\infty}^{+\infty} g(x) \cdot f(x) \, dx
\]

\subsection{Variance et Écart-type}

La variance de \( X \) est définie par :

\[
\text{Var}(X) = \mathbb{E}[(X - \mathbb{E}[X])^2]
\]

\paragraph{Cas Discret}
\[
\text{Var}(X) = \sum_{i} (x_i - \mathbb{E}[X])^2 \cdot p_i
\]

\paragraph{Cas Continu}
\[
\text{Var}(X) = \int_{-\infty}^{+\infty} (x - \mathbb{E}[X])^2 \cdot f(x) \, dx
\]

L’écart-type est donné par : \( \sigma(X) = \sqrt{\text{Var}(X)} \).

\subsection{Fonction Génératrice des Moments (MGF)}

La fonction génératrice des moments de \( X \), notée \( M_X(t) \), est définie par :

\paragraph{Cas Discret}
\[
M_X(t) = \mathbb{E}[e^{tX}] = \sum_{i} e^{t x_i} \cdot p_i
\]

\paragraph{Cas Continu}
\[
M_X(t) = \mathbb{E}[e^{tX}] = \int_{-\infty}^{+\infty} e^{t x} \cdot f(x) \, dx
\]

\subsection{Fonction Caractéristique}

La fonction caractéristique de \( X \), notée \( \phi_X(t) \), est définie par :

\paragraph{Cas Discret}
\[
\phi_X(t) = \mathbb{E}[e^{itX}] = \sum_{i} e^{it x_i} \cdot p_i
\]

\paragraph{Cas Continu}
\[
\phi_X(t) = \mathbb{E}[e^{itX}] = \int_{-\infty}^{+\infty} e^{it x} \cdot f(x) \, dx
\]

\newpage
\section{Exercices}


\exop{}{
Nous cherchons à calculer :
\[
\int_0^{+\infty} \frac{\cos(t)}{(1+t^2)}
\]
Pour ce faire nous considérons la fonction :
\[
f(x) = \frac{\cos(xt)}{(1+t^2)}
\]
Et la fonction :
\[
h(x) = \frac{t\sin(xt)}{(1+t^2)^2}
\]
\begin{enumerate}
    \item Montrer les fonctions sont bien définies sur $\RR$. et que la fonction $f$ est paire et la fonction $h$ est impaire. 
    \item Montrer que : 
    \[ 
        \forall x\in\RR : xf(x) = 2h(x)
    \]
    \item Montre que la fonction $f$ est continue et borné sur $\RR$.
    \item Montre que la fonction $h$ est de classe $\class^1$ sur $]0,+\infty[$, puis établir :
    \[\forall x \in\RR \quad : h^\prime(x) = f(x) - \int_{0}^{+\infty} \frac{\cos(xt)}{(1+t^2)^2}dt  \]
    \item Notons $k(x)=\int_{0}^{+\infty}\frac{\cos(xt)}{(1+t^2)}dt$, montre que $k$ est de classe $\class^1$, et que :
    \[
    k^\prime(x) = -h(x)
    \]
    \item En déduire que $f$ est de classe $\class^2$ sur $]0,+\infty[$ et que : 
    \[f^{\prime\prime}(x) = f(x) \quad \forall x\in ]0,+\infty[\]
    Et puis montrer que : 
    \[
    f(x) = \frac{\pi}{2}e^{-|x|} \quad \forall x\in \RR
    \]
    \item Conclure la valeur de l'intégrale.
\end{enumerate}
}{
\begin{enumerate}
    \item Soit, $x\in \RR$, on a pour tout $t>0$ :  $\left|\frac{\cos(xt)}{(1+t^2)}\right|\leq \frac{1}{1+t^2}$ ainsi comme $\int_0^{+\infty}  \frac{1}{1+t^2}$ converge donc $\int_0^{+\infty} \frac{\cos(xt)}{(1+t^2)}$ et par conséquence $f(x)$ est bien définie.  même chose pour la fonction $h$. 
    \item C'est facile à le montrer, déjà on a l'hypothèse de dominance vérifié :
    $$\left|\frac{\cos(xt)}{(1+t^2)}\right|\leq \frac{1}{1+t^2}$$
    De plus, la bornitude se découle aisément par  :
    \[ \forall x\in\RR \quad : 
    \left| \int_0^{+\infty} \frac{\cos(xt)}{(1+t^2)} dt\right| \leq \int_0^{+\infty} \frac{1}{(1+t^2)} dt
    \]
    \item  Soit $x \in \mathbb{R}$. On a, à l'aide d'une intégration par parties, pour tout $T$ de $[0 ;+\infty[$ :
    $$
    x \int_0^T \frac{\cos (x t)}{1+t^2} \mathrm{~d} t=\frac{\sin x T}{1+T^2}+2 \int_0^T \frac{t \sin (x t)}{\left(1+t^2\right)^2} \mathrm{~d} t
    $$ d'où, en faisant tendre $T$ vers $+\infty: x f(x)=2 h(x)$.
    \item Montrons que la fonction $h$ est de classe $\class^1$, notons $\phi(x,t)=\frac{t\sin(xt)}{(1+t^2)^2}$ on a :
        \begin{itemize}
        \item  \textbf{Dérivabilité}  \\  
        La fonction $x \mapsto \phi(x,t)$ est dérivable et de classe $\class^{1}$ sur $]0,+\infty[$ pour tout $t\in ]0,+\infty[$, ainsi la dérivée par rapport au $x$ sont :
        \[
            \frac{\partial }{\partial x}\phi(x,t) = \frac{t^2\cos(xt)}{(1+t^2)^2} 
        \]
        \item  \textbf{Continuité par morceau}  \\ 
        Pour tout réel  $x\in ]0,+\infty[$, les deux fonctions $t\mapsto \phi(x,t)$  et $t\mapsto \frac{\partial \phi}{\partial x}(x,t) $ sont continues par morceaux sur $]0,+\infty[$.
        \item  \textbf{Intégrabilité}  \\  
        Pour tout $x\in ]0,+\infty[$ la fonction $t \mapsto \phi(x,t)$  est intégrable sur $]0,+\infty[$.
        \item  \textbf{Hypothèse de domination}  \\  
        Il existe une fonction $\psi(t)=\frac{t^2}{(1+t^2)^2}$ définit et intégrable sur $]0,+\infty[$, et on a: 
        \[ \left|\frac{\partial \phi}{\partial x}(x,t)\right| \leq \psi(t) \quad\forall (x,t)\in  ]0,+\infty[\times  ]0,+\infty[ \]
        \end{itemize}
        Donc on conclut que, $h$ est de classe $\class^1$ sur $]0,+\infty[$ et on a :
    \cc{
    h^{\prime}(x)=\int_{0}^{+\infty} \frac{t^{2} \cos(xt)} {(1+t^{2})^{2}} \; dt = \int_{0}^{+\infty}\frac{\cos(xt)}{(1+t^{2})}dt -  \int_{0}^{+\infty}\frac{\cos(xt)} {(1+t^{2})^{2}}dt = f(x) - k(x)
    }
  \item La même argumentation s'applique à $k: x \longmapsto \int_0^{+\infty} \frac{\cos (x t)}{\left(1+t^2\right)^2} \mathrm{~d} t$, et montre que $k$ est de classe $\class^1$ sur $[0 ;+\infty[$ et que :
    \cc{
    \forall x \in[0 ;+\infty[, k^{\prime}(x)=-\int_0^{+\infty} \frac{t \sin (x t)}{\left(1+t^2\right)^2} \mathrm{~d} t=-h(x)
    }
    \item On a : $f(x) = \frac{1}{x}h(x)$ donc $f$ est de classe $\class C^1$ sur $]0,+\infty[$ donc $h^\prime(x)=f(x)-k(x)$ est de classe $\class^1$ et par suite $h$ et $f$ sont de classe $\class^2$ sur $]0,+\infty[$. On déduit finalement $ \forall x \in] 0 ;+\infty[$: 
    \begin{align*}
    xf(x) &= 2h(x) \\ 
 x f^{\prime}(x)+f(x)&=2 h^{\prime}(x) \\ 
 x f^{\prime}(x)+f(x)&= 2 f(x)-2 k(x)  \\ 
      x f^{\prime}(x) & = f(x)- 2k(x)  \\
      x f^{\prime\prime}(x)+f^{\prime}(x) & = f^{\prime}(x)- 2k^{\prime}(x)  \\
      x f^{\prime\prime}(x) & =2h(x)  \\
      x f^{\prime\prime}(x) & =xf(x)  
    \end{align*}
Ce qui montre que :
\cc{
f^{\prime\prime}(x)  = f(x) 
}
\item L’équation différentielle est de second ordre, ses solutions sont de la forme :
\[
\forall x\in]0,+\infty[ \quad : f(x) = C_0e^{x} + C_1e^{-x}
\]
Or puisque la fonction est continue sur $\RR$ et donc en $0$ alors :
\[
\forall x\in[0,+\infty[ \quad : f(x) = C_0e^{x} + C_1e^{-x}
\]
De plus, puisque la fonction est borné alors $C_0=0$, Donc :
\[
f(x) = f(0)e^{-x}
\]
On la condition initiale :
\[f(0) = \int_0^{+\infty} \frac{1}{1+t^2}dt = \frac{\pi}{2}\]
Donc, $f(x) = \frac{\pi}{2}e^{-x}$ pour tout $x\in[0,+\infty[$ et finalement par parité de la fonction on déduit que :
\cc{
\forall x\in\RR \quad : f(x) = \frac{\pi}{2}e^{-|x|}
}
On peut maintenant déduire que :
\cc{
\int_0^{+\infty} \frac{\cos(t)}{1+t^2} dt = \frac{\pi}{2}
}
\end{enumerate}
}


\exop{}{
On cherche à évaluer l'intégrale :
\[
\int_0^{+\infty} \frac{\sin t}{t}dt 
\]

    On note $f(x)=\int_0^{+\infty} \frac{1-\mathrm{e}^{-x t}}{t} \sin t \mathrm{~d} t$, nous cherchons à évaluer cette fonction pour tout $x$.
\begin{enumerate}
    \item Montrer que la fonction est bien définie, càd  pour tout $x$ de $[0 ;+\infty[$, l'intégrale impropre $\int_0^{+\infty} \frac{1-\mathrm{e}^{-x t}}{t} \sin t \mathrm{~d} t$ converge ;
    \item  Montrer que $f$ est de classe $C^1$ sur $] 0 ;+\infty[$ et :
        $$
        \forall x \in] 0 ;+\infty[, f^{\prime}(x)=\frac{1}{1+x^2}.
        $$
    \item En déduire qu'il existe $C \in \mathbb{R}$ tel que : $\forall x \in] 0 ;+\infty[$,
    $$
    f(x)=\operatorname{Arctan} x+C .
    $$
    \item Montre que pour tout $x>0$:
    \[0\leq f(x)-f(0) \leq 2x\]
    \item Montre que $f$ est continue en $0$, puis déduire $C$.
    \item En déduire la valeur de l'intégrale objectif de l'exercice. 
\end{enumerate}
}{
\begin{enumerate}
    \item Soit $x \in\left[0 ;+\infty\left[\right.\right.$. L'application $t \longmapsto \frac{1-\mathrm{e}^{-x t}}{t} \sin t$ est continue sur $] 0 ;+\infty\left[, \frac{1-\mathrm{e}^{-x t}}{t} \sin t \xrightarrow[t \rightarrow 0]{\longrightarrow} 0 ;\right.$ comme $\frac{1-\mathrm{e}^{-x t}}{t} \sin t=\frac{\sin t}{t}-\frac{\sin t}{t} \mathrm{e}^{-x t}$, que l'intégrale impropre $\int_1^{\rightarrow+\infty} \frac{\sin t}{t} \mathrm{~d} t$ converge, et que $($ si $x>0) t \longmapsto \frac{\sin t}{t} \mathrm{e}^{-x t}$ est intégrable sur $[1 ;+\infty[$, l'intégrale impropre $\int_0^{\rightarrow+\infty} \frac{1-\mathrm{e}^{-x t}}{t} \sin t \mathrm{~d} t$ converge.

    \item Montrons que la fonction $f$ est de classe $\class^1$ sur $]0,+\infty[$, il suffit d'étudier : $x\mapsto  \int_0^{+\infty} \frac{e^{-x t}\sin(t)}{t}dt$, sur des intervalles $[a,+\infty[$ pour tout $a>0$. \\
    Soit $a>0$, notons $\phi(x,t)=\frac{e^{-x t}\sin(t)}{t}$ on a : 
        \begin{itemize}
        \item  \textbf{Dérivabilité}  \\  
        La fonction $x \mapsto \phi(x,t)$ est dérivable et de classe $\class^{1}$ sur $[a,+\infty[$ pour tout $t\in ]0,+\infty[$, ainsi la dérivée par rapport au $x$ sont :
        \[
            \frac{\partial }{\partial x}\phi(x,t) = -e^{-xt}\sin(t) 
        \]
        \item  \textbf{Continuité par morceau}  \\ 
        Pour tout réel  $x\in [a,+\infty[$, les deux fonctions $t\mapsto \phi(x,t)$  et $t\mapsto \frac{\partial \phi}{\partial x}(x,t) $ sont continues par morceaux sur $]0,+\infty[$.
        \item  \textbf{Intégrabilité}  \\  
        Pour tout $x\in [a,+\infty[$ la fonction $t \mapsto \phi(x,t)$  est intégrable sur $]0,+\infty[$.
        \item  \textbf{Hypothèse de domination}  \\  
        Il existe une fonction $\psi(t)=e^{-at}$ définit et intégrable sur $]0,+\infty[$, et on a: 
        \[ \left|\frac{\partial \phi}{\partial x}(x,t)\right| \leq \psi(t) \quad\forall (x,t)\in  [a,+\infty[\times  ]0,+\infty[ \]
        \end{itemize}
        Donc on conclut que, $h$ est de classe $\class^1$ sur $[a,+\infty[$ et on a :
    \[f^{\prime}(x)=\int_{0}^{+\infty} \sin(t)e^{-xt}dt = \Im\left( \int_{0}^{+\infty} e^{-(x-i)t}dt \right) =  \Im\left( \frac{1}{x-i} \right) = \frac{1}{1+x^2}\]
    
    Cette propriété, s’étendre sur tout l'intervalle $]0,+\infty[$ car $\bigcup_{a>0} [a,+\infty[ = ]0,+\infty[$ et donc : 
    \cc{ \forall x\in]0,+\infty[ \quad: f^{\prime}(x) = \frac{1}{1+x^2} }

    \item Comme $f^\prime(x) = \frac{1}{1+x^2}$ Donc $\exists C\in\RR$ tq :
    \cc{
    f(x) = \arctan(x) + C \quad\forall x>0
    }

    % montre que f(x)-f(0) < 2x
    \item On peut montrer aisément que :
    \[
        A(t) = -\sum_{n=0}^{+\infty} \frac{(-1)^n}{(n+1)!}t^n
    \]
    Et donc la fonction est de classe $\class^{\infty}$ de plus sa dérivé premier est : 
    \[
    A^\prime(t) = \frac{(1+t)e^{-t}-1}{t^2} \leq 0 \quad \forall t>0
    \]
    Puis,  Soit $x \in] 0 ;+\infty[$. Par le changement de variable $u=x t$ on a :
    $$
    \begin{aligned}
    f(x) & =\int_0^{+\infty} \frac{1-\mathrm{e}^{-u}}{u} \sin \frac{u}{x} \mathrm{~d} u \\
    & =\int_0^{+\infty} A(u) \sin \frac{u}{x} \mathrm{~d} u.
    \end{aligned}
    $$
    On obtient, à l'aide d'une intégration par parties, pour tout $U$ de $[0 ;+\infty[$ :
    $$
    \int_0^U A(u) \cos \frac{u}{x} \mathrm{~d} u  =x-x A(U) \cos \frac{U}{x}+x \int_0^U A^{\prime}(u) \cos \frac{u}{x} \mathrm{~d} u .
    $$
    D'une part : $x A(U) \cos \frac{U}{x} \xrightarrow[U \rightarrow+\infty]{ } 0, \operatorname{car} A(U) \xrightarrow[U \rightarrow+\infty]{}$.
    D'autre part, comme $A^{\prime} \leqslant 0$ :
    $$
    \begin{aligned}
    \left|\int_0^U A^{\prime}(u) \cos \frac{u}{x} \mathrm{~d} u\right| \leqslant & \int_0^U\left(-A^{\prime}(u)\right)\left|\cos \frac{u}{x}\right| \mathrm{d} u \\
    & \leqslant \int_0^U-\left(A^{\prime}(u)\right) \mathrm{d} u=1-A(U)
    \end{aligned}
    $$
On en déduit, en faisant tendre  $U$ vers : $+\infty$ : \\
\[
\begin{aligned}
0 \leqslant f(x)-f(0) & =f(x) \\
& =\int_0^{+\infty} A(u) \cos \frac{u}{x} \mathrm{~d} u \leqslant 2 x.
\end{aligned}
\]

    % Montre que f est continue en 0, puis déduire C
    \item Puisque : $0\leq f(x) - f(0) \leq 2x$ alors en tendant $x\rightarrow 0$ on aurait la continuité en $0$. Et par conséquence $C=0$ donc :
    \cc{
    f(x) = \arctan(x) \quad\forall x>0
    }

    % déduire la valeur de integrale 
    \item On a, pour tout $x$ de $] 0 ;+\infty[$ :
    $$
    \begin{aligned}
    \left|\int_0^{+\infty} \frac{\mathrm{e}^{-x t} \sin t}{t} \mathrm{~d} t\right| & \leqslant \int_0^{+\infty} \mathrm{e}^{-x t}\left|\frac{\sin t}{t}\right| \mathrm{d} t \\
    & \leqslant \int_0^{+\infty} \mathrm{e}^{-x t} \mathrm{~d} t=\frac{1}{x}
    \end{aligned}
    $$
Donc $f(x) \underset{x \rightarrow+\infty}{\longrightarrow} \int_0^{+\infty} \frac{\sin t}{t} \mathrm{~d} t$.
D'autre part : $f(x)=\operatorname{Arctan} x \underset{x \rightarrow+\infty}{\longrightarrow} \frac{\pi}{2}$.
On conclut : 
\cc{\int_0^{+\infty} \frac{\sin t}{t} \mathrm{~d} t=\frac{\pi}{2}}
\end{enumerate}
}



\end{document}